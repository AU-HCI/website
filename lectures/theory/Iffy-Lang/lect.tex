\documentclass{article}
\usepackage{parskip}
\usepackage{amsmath,amssymb,amsthm}
\usepackage{fullpage}
\usepackage{hyperref}
\usepackage{color}
\usepackage{mathpartir}
\usepackage{qtree}

% generated by Ott 0.31 from: Iffy.ott
\newcommand{\ottdrule}[4][]{{\displaystyle\frac{\begin{array}{l}#2\end{array}}{#3}\quad\ottdrulename{#4}}}
\newcommand{\ottusedrule}[1]{\[#1\]}
\newcommand{\ottpremise}[1]{ #1 \\}
\newenvironment{ottdefnblock}[3][]{ \framebox{\mbox{#2}} \quad #3 \\[0pt]}{}
\newenvironment{ottfundefnblock}[3][]{ \framebox{\mbox{#2}} \quad #3 \\[0pt]\begin{displaymath}\begin{array}{l}}{\end{array}\end{displaymath}}
\newcommand{\ottfunclause}[2]{ #1 \equiv #2 \\}
\newcommand{\ottnt}[1]{\mathit{#1}}
\newcommand{\ottmv}[1]{\mathit{#1}}
\newcommand{\ottkw}[1]{\mathbf{#1}}
\newcommand{\ottsym}[1]{#1}
\newcommand{\ottcom}[1]{\text{#1}}
\newcommand{\ottdrulename}[1]{\textsc{#1}}
\newcommand{\ottcomplu}[5]{\overline{#1}^{\,#2\in #3 #4 #5}}
\newcommand{\ottcompu}[3]{\overline{#1}^{\,#2<#3}}
\newcommand{\ottcomp}[2]{\overline{#1}^{\,#2}}
\newcommand{\ottgrammartabular}[1]{\begin{supertabular}{llcllllll}#1\end{supertabular}}
\newcommand{\ottmetavartabular}[1]{\begin{supertabular}{ll}#1\end{supertabular}}
\newcommand{\ottrulehead}[3]{$#1$ & & $#2$ & & & \multicolumn{2}{l}{#3}}
\newcommand{\ottprodline}[6]{& & $#1$ & $#2$ & $#3 #4$ & $#5$ & $#6$}
\newcommand{\ottfirstprodline}[6]{\ottprodline{#1}{#2}{#3}{#4}{#5}{#6}}
\newcommand{\ottlongprodline}[2]{& & $#1$ & \multicolumn{4}{l}{$#2$}}
\newcommand{\ottfirstlongprodline}[2]{\ottlongprodline{#1}{#2}}
\newcommand{\ottbindspecprodline}[6]{\ottprodline{#1}{#2}{#3}{#4}{#5}{#6}}
\newcommand{\ottprodnewline}{\\}
\newcommand{\ottinterrule}{\\[5.0mm]}
\newcommand{\ottafterlastrule}{\\}
\newcommand{\ottmetavars}{
\ottmetavartabular{
 $ \ottmv{term\_var} ,\, \ottmv{x} ,\, \ottmv{y} ,\, \ottmv{z} ,\, \ottmv{f} ,\, \ottmv{u} ,\, \ottmv{v} ,\, \ottmv{s} ,\, \ottmv{p} ,\, \ottmv{w} ,\, \ottmv{r} ,\, \mathsf{not} ,\, \ottmv{NAME} ,\, \mathsf{NEWLINE} $ &  \\
}}

\newcommand{\ottprogram}{
\ottrulehead{\ottnt{program}  ,\ \ottnt{P}}{::=}{}\ottprodnewline
\ottfirstprodline{|}{\ottmv{x}  \ottsym{:=}  \ottnt{b}}{}{}{}{}\ottprodnewline
\ottprodline{|}{\ottnt{P_{{\mathrm{1}}}} \, \mathsf{NEWLINE} \, \ottnt{P_{{\mathrm{2}}}}}{}{}{}{}}

\newcommand{\ottbooleans}{
\ottrulehead{\ottnt{booleans}  ,\ \ottnt{b}}{::=}{}\ottprodnewline
\ottfirstprodline{|}{\ottmv{x}}{}{}{}{}\ottprodnewline
\ottprodline{|}{ 0 }{}{}{}{}\ottprodnewline
\ottprodline{|}{ 1 }{}{}{}{}\ottprodnewline
\ottprodline{|}{ \ottnt{b_{{\mathrm{1}}}}  \land  \ottnt{b_{{\mathrm{2}}}} }{}{}{}{}\ottprodnewline
\ottprodline{|}{ \ottnt{b_{{\mathrm{1}}}}  \lor  \ottnt{b_{{\mathrm{2}}}} }{}{}{}{}\ottprodnewline
\ottprodline{|}{ \mathsf{if}\, \ottnt{b} \,\mathsf{then}\, \ottnt{b_{{\mathrm{1}}}} \,\mathsf{else}\, \ottnt{b_{{\mathrm{2}}}} }{}{}{}{}\ottprodnewline
\ottprodline{|}{ \mathsf{fun}\, \ottmv{x}  \Rightarrow  \ottnt{b} }{}{}{}{}\ottprodnewline
\ottprodline{|}{ \mathsf{let}\, \ottmv{x}  =  \ottnt{b_{{\mathrm{1}}}} \,\mathsf{in}\, \ottnt{b_{{\mathrm{2}}}} }{}{}{}{}\ottprodnewline
\ottprodline{|}{ \mathsf{app}\, \ottnt{b_{{\mathrm{1}}}} \, \ottnt{b_{{\mathrm{2}}}} }{}{}{}{}\ottprodnewline
\ottprodline{|}{\ottsym{(}  \ottnt{b}  \ottsym{)}}{}{}{}{}}

\newcommand{\ottgrammar}{\ottgrammartabular{
\ottprogram\ottinterrule
\ottbooleans\ottafterlastrule
}}

% defnss
% defns Opsem
%% defn red
\newcommand{\ottdruleBetaName}[0]{\ottdrulename{Beta}}
\newcommand{\ottdruleBeta}[1]{\ottdrule[#1]{%
}{
  \mathsf{app}\, \ottsym{(}   \mathsf{fun}\, \ottmv{x}  \Rightarrow  \ottnt{b}   \ottsym{)} \, \ottnt{b'}   \rightsquigarrow  \ottsym{[}  \ottnt{b'}  \ottsym{/}  \ottmv{x}  \ottsym{]}  \ottnt{b} }{%
{\ottdrulename{Beta}}{}%
}}


\newcommand{\ottdruleFunName}[0]{\ottdrulename{Fun}}
\newcommand{\ottdruleFun}[1]{\ottdrule[#1]{%
\ottpremise{ \ottnt{b}  \rightsquigarrow  \ottnt{b'} }%
}{
 \ottsym{(}   \mathsf{fun}\, \ottmv{x}  \Rightarrow  \ottnt{b}   \ottsym{)}  \rightsquigarrow  \ottsym{(}   \mathsf{fun}\, \ottmv{x}  \Rightarrow  \ottnt{b'}   \ottsym{)} }{%
{\ottdrulename{Fun}}{}%
}}


\newcommand{\ottdruleAppOneName}[0]{\ottdrulename{App1}}
\newcommand{\ottdruleAppOne}[1]{\ottdrule[#1]{%
\ottpremise{ \ottnt{b_{{\mathrm{1}}}}  \rightsquigarrow  \ottnt{b'_{{\mathrm{1}}}} }%
}{
  \mathsf{app}\, \ottnt{b_{{\mathrm{1}}}} \, \ottnt{b_{{\mathrm{2}}}}   \rightsquigarrow   \mathsf{app}\, \ottnt{b'_{{\mathrm{1}}}} \, \ottnt{b_{{\mathrm{2}}}}  }{%
{\ottdrulename{App1}}{}%
}}


\newcommand{\ottdruleAppTwoName}[0]{\ottdrulename{App2}}
\newcommand{\ottdruleAppTwo}[1]{\ottdrule[#1]{%
\ottpremise{ \ottnt{b_{{\mathrm{2}}}}  \rightsquigarrow  \ottnt{b'_{{\mathrm{2}}}} }%
}{
  \mathsf{app}\, \ottnt{b_{{\mathrm{1}}}} \, \ottnt{b_{{\mathrm{2}}}}   \rightsquigarrow   \mathsf{app}\, \ottnt{b_{{\mathrm{1}}}} \, \ottnt{b'_{{\mathrm{2}}}}  }{%
{\ottdrulename{App2}}{}%
}}


\newcommand{\ottdruleAndTrueName}[0]{\ottdrulename{AndTrue}}
\newcommand{\ottdruleAndTrue}[1]{\ottdrule[#1]{%
}{
 \ottsym{(}    1   \land   1    \ottsym{)}  \rightsquigarrow   1  }{%
{\ottdrulename{AndTrue}}{}%
}}


\newcommand{\ottdruleAndFalseOneName}[0]{\ottdrulename{AndFalse1}}
\newcommand{\ottdruleAndFalseOne}[1]{\ottdrule[#1]{%
}{
 \ottsym{(}    0   \land   1    \ottsym{)}  \rightsquigarrow   0  }{%
{\ottdrulename{AndFalse1}}{}%
}}


\newcommand{\ottdruleAndFalseTwoName}[0]{\ottdrulename{AndFalse2}}
\newcommand{\ottdruleAndFalseTwo}[1]{\ottdrule[#1]{%
}{
 \ottsym{(}    1   \land   0    \ottsym{)}  \rightsquigarrow   0  }{%
{\ottdrulename{AndFalse2}}{}%
}}


\newcommand{\ottdruleAndFalseName}[0]{\ottdrulename{AndFalse}}
\newcommand{\ottdruleAndFalse}[1]{\ottdrule[#1]{%
}{
 \ottsym{(}    0   \land   0    \ottsym{)}  \rightsquigarrow   0  }{%
{\ottdrulename{AndFalse}}{}%
}}


\newcommand{\ottdruleAndOneName}[0]{\ottdrulename{And1}}
\newcommand{\ottdruleAndOne}[1]{\ottdrule[#1]{%
\ottpremise{ \ottnt{b_{{\mathrm{1}}}}  \rightsquigarrow  \ottnt{b'_{{\mathrm{1}}}} }%
}{
 \ottsym{(}   \ottnt{b_{{\mathrm{1}}}}  \land  \ottnt{b_{{\mathrm{2}}}}   \ottsym{)}  \rightsquigarrow  \ottsym{(}   \ottnt{b'_{{\mathrm{1}}}}  \land  \ottnt{b_{{\mathrm{2}}}}   \ottsym{)} }{%
{\ottdrulename{And1}}{}%
}}


\newcommand{\ottdruleAndTwoName}[0]{\ottdrulename{And2}}
\newcommand{\ottdruleAndTwo}[1]{\ottdrule[#1]{%
\ottpremise{ \ottnt{b_{{\mathrm{2}}}}  \rightsquigarrow  \ottnt{b'_{{\mathrm{2}}}} }%
}{
 \ottsym{(}   \ottnt{b_{{\mathrm{1}}}}  \land  \ottnt{b_{{\mathrm{2}}}}   \ottsym{)}  \rightsquigarrow  \ottsym{(}   \ottnt{b_{{\mathrm{1}}}}  \land  \ottnt{b'_{{\mathrm{2}}}}   \ottsym{)} }{%
{\ottdrulename{And2}}{}%
}}


\newcommand{\ottdruleOrTrueName}[0]{\ottdrulename{OrTrue}}
\newcommand{\ottdruleOrTrue}[1]{\ottdrule[#1]{%
}{
 \ottsym{(}    1   \lor   1    \ottsym{)}  \rightsquigarrow   1  }{%
{\ottdrulename{OrTrue}}{}%
}}


\newcommand{\ottdruleOrTrueTwoName}[0]{\ottdrulename{OrTrue2}}
\newcommand{\ottdruleOrTrueTwo}[1]{\ottdrule[#1]{%
}{
 \ottsym{(}    0   \lor   1    \ottsym{)}  \rightsquigarrow   1  }{%
{\ottdrulename{OrTrue2}}{}%
}}


\newcommand{\ottdruleOrTrueOneName}[0]{\ottdrulename{OrTrue1}}
\newcommand{\ottdruleOrTrueOne}[1]{\ottdrule[#1]{%
}{
 \ottsym{(}    1   \lor   0    \ottsym{)}  \rightsquigarrow   1  }{%
{\ottdrulename{OrTrue1}}{}%
}}


\newcommand{\ottdruleOrFalseName}[0]{\ottdrulename{OrFalse}}
\newcommand{\ottdruleOrFalse}[1]{\ottdrule[#1]{%
}{
 \ottsym{(}    0   \lor   0    \ottsym{)}  \rightsquigarrow   0  }{%
{\ottdrulename{OrFalse}}{}%
}}


\newcommand{\ottdruleOrOneName}[0]{\ottdrulename{Or1}}
\newcommand{\ottdruleOrOne}[1]{\ottdrule[#1]{%
\ottpremise{ \ottnt{b_{{\mathrm{1}}}}  \rightsquigarrow  \ottnt{b'_{{\mathrm{1}}}} }%
}{
 \ottsym{(}   \ottnt{b_{{\mathrm{1}}}}  \lor  \ottnt{b_{{\mathrm{2}}}}   \ottsym{)}  \rightsquigarrow  \ottsym{(}   \ottnt{b'_{{\mathrm{1}}}}  \lor  \ottnt{b_{{\mathrm{2}}}}   \ottsym{)} }{%
{\ottdrulename{Or1}}{}%
}}


\newcommand{\ottdruleOrTwoName}[0]{\ottdrulename{Or2}}
\newcommand{\ottdruleOrTwo}[1]{\ottdrule[#1]{%
\ottpremise{ \ottnt{b_{{\mathrm{2}}}}  \rightsquigarrow  \ottnt{b'_{{\mathrm{2}}}} }%
}{
 \ottsym{(}   \ottnt{b_{{\mathrm{1}}}}  \lor  \ottnt{b_{{\mathrm{2}}}}   \ottsym{)}  \rightsquigarrow  \ottsym{(}   \ottnt{b_{{\mathrm{1}}}}  \lor  \ottnt{b'_{{\mathrm{2}}}}   \ottsym{)} }{%
{\ottdrulename{Or2}}{}%
}}


\newcommand{\ottdruleIfTrueName}[0]{\ottdrulename{IfTrue}}
\newcommand{\ottdruleIfTrue}[1]{\ottdrule[#1]{%
}{
  \mathsf{if}\,  1  \,\mathsf{then}\, \ottnt{b_{{\mathrm{1}}}} \,\mathsf{else}\, \ottnt{b_{{\mathrm{2}}}}   \rightsquigarrow  \ottnt{b_{{\mathrm{1}}}} }{%
{\ottdrulename{IfTrue}}{}%
}}


\newcommand{\ottdruleIfFalseName}[0]{\ottdrulename{IfFalse}}
\newcommand{\ottdruleIfFalse}[1]{\ottdrule[#1]{%
}{
  \mathsf{if}\,  0  \,\mathsf{then}\, \ottnt{b_{{\mathrm{1}}}} \,\mathsf{else}\, \ottnt{b_{{\mathrm{2}}}}   \rightsquigarrow  \ottnt{b_{{\mathrm{2}}}} }{%
{\ottdrulename{IfFalse}}{}%
}}


\newcommand{\ottdruleIfOneName}[0]{\ottdrulename{If1}}
\newcommand{\ottdruleIfOne}[1]{\ottdrule[#1]{%
\ottpremise{ \ottnt{b}  \rightsquigarrow  \ottnt{b'} }%
}{
  \mathsf{if}\, \ottnt{b} \,\mathsf{then}\, \ottnt{b_{{\mathrm{1}}}} \,\mathsf{else}\, \ottnt{b_{{\mathrm{2}}}}   \rightsquigarrow   \mathsf{if}\, \ottnt{b'} \,\mathsf{then}\, \ottnt{b_{{\mathrm{1}}}} \,\mathsf{else}\, \ottnt{b_{{\mathrm{2}}}}  }{%
{\ottdrulename{If1}}{}%
}}


\newcommand{\ottdruleIfTwoName}[0]{\ottdrulename{If2}}
\newcommand{\ottdruleIfTwo}[1]{\ottdrule[#1]{%
\ottpremise{ \ottnt{b_{{\mathrm{1}}}}  \rightsquigarrow  \ottnt{b'_{{\mathrm{1}}}} }%
}{
  \mathsf{if}\, \ottnt{b} \,\mathsf{then}\, \ottnt{b_{{\mathrm{1}}}} \,\mathsf{else}\, \ottnt{b_{{\mathrm{2}}}}   \rightsquigarrow   \mathsf{if}\, \ottnt{b} \,\mathsf{then}\, \ottnt{b'_{{\mathrm{1}}}} \,\mathsf{else}\, \ottnt{b_{{\mathrm{2}}}}  }{%
{\ottdrulename{If2}}{}%
}}


\newcommand{\ottdruleIfThreeName}[0]{\ottdrulename{If3}}
\newcommand{\ottdruleIfThree}[1]{\ottdrule[#1]{%
\ottpremise{ \ottnt{b_{{\mathrm{2}}}}  \rightsquigarrow  \ottnt{b'_{{\mathrm{2}}}} }%
}{
  \mathsf{if}\, \ottnt{b} \,\mathsf{then}\, \ottnt{b_{{\mathrm{1}}}} \,\mathsf{else}\, \ottnt{b_{{\mathrm{2}}}}   \rightsquigarrow   \mathsf{if}\, \ottnt{b} \,\mathsf{then}\, \ottnt{b_{{\mathrm{1}}}} \,\mathsf{else}\, \ottnt{b'_{{\mathrm{2}}}}  }{%
{\ottdrulename{If3}}{}%
}}

\newcommand{\ottdefnred}[1]{\begin{ottdefnblock}[#1]{$ \ottnt{b_{{\mathrm{1}}}}  \rightsquigarrow  \ottnt{b_{{\mathrm{2}}}} $}{}
\ottusedrule{\ottdruleBeta{}}
\ottusedrule{\ottdruleFun{}}
\ottusedrule{\ottdruleAppOne{}}
\ottusedrule{\ottdruleAppTwo{}}
\ottusedrule{\ottdruleAndTrue{}}
\ottusedrule{\ottdruleAndFalseOne{}}
\ottusedrule{\ottdruleAndFalseTwo{}}
\ottusedrule{\ottdruleAndFalse{}}
\ottusedrule{\ottdruleAndOne{}}
\ottusedrule{\ottdruleAndTwo{}}
\ottusedrule{\ottdruleOrTrue{}}
\ottusedrule{\ottdruleOrTrueTwo{}}
\ottusedrule{\ottdruleOrTrueOne{}}
\ottusedrule{\ottdruleOrFalse{}}
\ottusedrule{\ottdruleOrOne{}}
\ottusedrule{\ottdruleOrTwo{}}
\ottusedrule{\ottdruleIfTrue{}}
\ottusedrule{\ottdruleIfFalse{}}
\ottusedrule{\ottdruleIfOne{}}
\ottusedrule{\ottdruleIfTwo{}}
\ottusedrule{\ottdruleIfThree{}}
\end{ottdefnblock}}

%% defn mred
\newcommand{\ottdruleReflName}[0]{\ottdrulename{Refl}}
\newcommand{\ottdruleRefl}[1]{\ottdrule[#1]{%
}{
 \ottnt{b}  \rightsquigarrow^*  \ottnt{b} }{%
{\ottdrulename{Refl}}{}%
}}


\newcommand{\ottdruleStepName}[0]{\ottdrulename{Step}}
\newcommand{\ottdruleStep}[1]{\ottdrule[#1]{%
\ottpremise{ \ottnt{b_{{\mathrm{1}}}}  \rightsquigarrow  \ottnt{b_{{\mathrm{2}}}} }%
}{
 \ottnt{b_{{\mathrm{1}}}}  \rightsquigarrow^*  \ottnt{b_{{\mathrm{2}}}} }{%
{\ottdrulename{Step}}{}%
}}


\newcommand{\ottdruleMultName}[0]{\ottdrulename{Mult}}
\newcommand{\ottdruleMult}[1]{\ottdrule[#1]{%
\ottpremise{  \ottnt{b_{{\mathrm{1}}}}  \rightsquigarrow^*  \ottnt{b_{{\mathrm{2}}}}   \quad   \ottnt{b_{{\mathrm{2}}}}  \rightsquigarrow^*  \ottnt{b_{{\mathrm{3}}}}  }%
}{
 \ottnt{b_{{\mathrm{1}}}}  \rightsquigarrow^*  \ottnt{b_{{\mathrm{3}}}} }{%
{\ottdrulename{Mult}}{}%
}}

\newcommand{\ottdefnmred}[1]{\begin{ottdefnblock}[#1]{$ \ottnt{b_{{\mathrm{1}}}}  \rightsquigarrow^*  \ottnt{b_{{\mathrm{2}}}} $}{}
\ottusedrule{\ottdruleRefl{}}
\ottusedrule{\ottdruleStep{}}
\ottusedrule{\ottdruleMult{}}
\end{ottdefnblock}}


\newcommand{\ottdefnsOpsem}{
\ottdefnred{}\ottdefnmred{}}

\newcommand{\ottdefnss}{
\ottdefnsOpsem
}

\newcommand{\ottall}{\ottmetavars\\[0pt]
\ottgrammar\\[5.0mm]
\ottdefnss}


% Theorems
\newtheorem{theorem}{Theorem}
\newtheorem{lemma}[theorem]{Lemma}
\newtheorem{fact}[theorem]{Fact}
\newtheorem{corollary}[theorem]{Corollary}
\newtheorem{definition}[theorem]{Definition}
\newtheorem{example}[theorem]{Example}
\newtheorem{remark}[theorem]{Remark}
\newtheorem{proposition}[theorem]{Proposition}
\newtheorem{notn}[theorem]{Notation}
\newtheorem{observation}[theorem]{Observation}

% Commands that are useful for writing about type theory and programming language design.
%% \newcommand{\case}[4]{\text{case}\ #1\ \text{of}\ #2\text{.}#3\text{,}#2\text{.}#4}
\newcommand{\interp}[1]{\llbracket #1 \rrbracket}
\newcommand{\normto}[0]{\rightsquigarrow^{!}}
\newcommand{\join}[0]{\downarrow}
\newcommand{\redto}[0]{\rightsquigarrow}
\newcommand{\nat}[0]{\mathbb{N}}
\newcommand{\fun}[2]{\lambda #1.#2}
\newcommand{\CRI}[0]{\text{CR-Norm}}
\newcommand{\CRII}[0]{\text{CR-Pres}}
\newcommand{\CRIII}[0]{\text{CR-Prog}}
\newcommand{\subexp}[0]{\sqsubseteq}
%% Must include \usepackage{mathrsfs} for this to work.
%% \newcommand{\powerset}[0]{\mathscr{P}}


\title{Lecture: Syntax, Inference Rules, and Our First Example
  PL\\Programming Languages (CSCI 3300)\vspace{-22px}}
\author{Prof. Harley Eades (heades@gru.edu).}  \date{\vspace{-22px}}
\begin{document}
\maketitle  

\section{Introduction to the course}
\label{sec:introduction}
Programming languages (PLs) are ubiquitous in computer science.  Any
student of computation has used some programming language whether they
a computer scientist, software engineer, or information
technologist. This then implies that the PLs are the most important
artifacts in the computational sciences.

One would find it hard to argue that PLs are important, but one could
argue that we already have a lot of PLs, and it seems as if these are
enough, but is this true?  I claim that it is not the case.
Computational devices are ubiquitous in our society.  They power our
power plants, our water supply, virtually every consumer product is
built using some computational device, in addition they power our
automobiles, our air planes, and our medical devies.  It is safe to
say that we put our vary lives in the hands of computational devices,
which implies that we put our lives in the hands of the programmers
who designed the software systems powering such devies.  What happens
when a software bug finds its way into the safety critical devices
that we depend on so dearly?

In 2010 there were a approximately a hundred reports made to the
National Highway Traffic Safety Administration of potential problems
with the braking system of the 2010 Toyota Prius \cite{Consumer:2010}.
The problem was that the anti-lock braking system would experience a
``short delay'' when the brakes where pressed by the driver of the
vehicle \cite{thedetroitbureau.com:2009}.  This actually caused some
crashes.  \textbf{Toyota found that this short delay was the result of
  a software bug, and was able to repair the the vehicles using a
  software update} \cite{Reuters:2009}.  Another incident where
substantial harm was caused was in 2002 where two planes collided over
\"{U}berlingen in Germany. A cargo plane operated by DHL collided with
a passenger flight holding fifty-one passengers.  Air-traffic control
did not notice the intersecting traffic until less than a minute
before the collision occurred.  Furthermore, the on-board collision
detection system did not alert the pilots until seconds before the
collision.  It was officially ruled by the German Federal Bureau of
Aircraft Accidents Investigation that the \textbf{on-board collision
  detection software was indeed faulty} \cite{Collision:2004}.

Avi Rubin is a professor at John Hopkins university and has recently
shown that pacemakers can be remotely hacked.  It may come to the
surprise to computer scientists, but pacemakers have wifi access now.
Avi has shown that this connection can be indeed hacked, and could be
used by the attacker to cause a fatal heart attack.  These hacks are
indeed due to software vulnerabilities -- that is bugs.  

How can we prevent these disastrous problems?  Currently, we using
testing.  Programmers come up with a huge number of test inputs to a
program, and then they run the program on each one of these tests.  If
the program behaves across all the tests, then we consider the program
as being correct.  Does this catch all bugs?  No!  In fact, no amount
of testing could ever \textbf{prove} the absence of all bugs!  If the
programmer is not clever enough, then they could indeed miss some
corner case.  This is evident in the examples above.  So is it
possible to be sure a program contains no bugs?  The only way to be
sure a program contains no bugs is by mathematically proving that it
meets some specification.  Every programmer uses some criteria to
write a program, and a programs specification corresponds to this
criteria transformed into a list of mathematical formulas.  Then if
one can prove that these formulas are all true with respect to the
program, then one can be sure with mathematical certainty that it is
correct.  That is bug free.  The joy of this approach is that if one
of the formulas cannot be proven, then a bug exists.  Furthermore,
this implies that the bug would be found during development instead of
in production.  So for example, the Toyota Prius bug in the breaking
system would have been found before all those crashes, and costly
recalls.  So why is it the case that companies use testing instead of
mathematical proof?  The reason is that current programming languages
are not amendable to mathematical proofs.  There are just to many
complex interactions.

Lets consider an example.  Suppose I say a function \verb|squared|
takes in an \verb|int| and outputs an \verb|int|, then we know
something about \verb|squared|, we know that if we take and apply it
to, say, \verb|2|, then we know that it will give us back an
\verb|int|, perhaps, \verb|4|.  Is it guaranteed that this is all it
does?  Suppose we are given the \verb|squared| function and it is
written in C\# do we know that it does nothing more than compute some
integer?  The answer is most definitely no.  In C\# the function
\verb|squared| could ask for some user input, print data to the
screen, and even make some network transactions or ask for the time of
day.  Hidden effects like these are all called \textbf{side effects}.
So in a programming language like C\#, C, C++, Java, PHP, Python, and
many more mainstream PLs we cannot really be sure what \verb|squared|
is doing.  Thus making it very difficult to reason about these
programs.  So can we fix this problem?  Can we design a PL that is
more amendable to mathematical reasoning?

I claim -- as do many other PL researchers -- that we can prevent
software bugs by mathematical reasoning by adopting new more
rigorously studied programming languages.  In fact, designing these
types of PLs is the very driving force of modern day PL research.  To
design these types of programming languages we must revisit the very
foundations of programming languages, and rebuild from the ground up.
It turns out that by adopting the notion of a pure strongly typed PL
we can rule out a large number of software bugs during development
without using testing, and before lives are taken!

A pure strongly typed PL is one in which there are either no side
effects at all or all side effects are accounted for in the type
system.  Consider the \verb|squared| function again.  The type of this
function can be written as $\verb|int| \to \verb|int|$ which means
taken an \verb|int| as input, and then return an \verb|int| as output.
Now suppose \verb|squared| prints out some data to the screen, then in
a pure strongly typed PL the type becomes $\verb|int| \to
\verb|IO(int)|$.  Here \verb|IO(int)| means the function returns an
\verb|int| and conducts some input-output effect.  This allows us to
take into account side effects when reasoning about the program.  So
how does all this work?  Well that is part of the topic of this
course.

Another part of the major topic of this course is on pure strongly
typed \textbf{functional} PLs.  Functional programming languages
differ from the PLs like C\# or Java by having a mathematical
foundation.  The sole datatype is a function, and in fact, the only
thing a functional PL contains are functions.  It turns out that we
can do a lot with this paradigm.  PLs like C\#, Java, and C are known
as imperative and object oriented languages.  These have a foundation
in an actual machine like a Turing Machine, while, functional PLs have
their foundation in mathematics.  However, it turns out that they are
equivalent in power.  One major part of this course is to understand
the design and implementation of functional PLs.  Functional
programming languages are also great for exploring different design
features, and we will use them to study features like parametric
polymorphism, and recursion.  By the end of this course each student
will understand how different each PL can be, and be able to better
differentiate between the various PLs when choosing one for a
particular project.

So a natural question is then is this all theoretical?  What baring
does any of this have on the real world?  A lot!  The industry is
already taking steps to push the old style of programming out, and
adopt new PLs that are highly motivated by the pure strongly typed
functional PLs.  A great example is Apple's new PL called Swift.  They
have decided to completely abandon all Objective C code from all their
IOS devices.  Swift is based on strongly typed functional programming
because of its brevity of code, and the strong correctness guarantees.
Another example, is Mozilla's Rust programming language which is a
systems programming language which is designed for speed.  A final
example is the Haskell programming language.  Haskell has paved the
way for functional programming language research and much of the
features of Swift and Rust where first studied in Haskell, and Haskell
is used in the industry, especially, the financial sector.  Lastly,
Haskell is the last major topic of this course.
% section introduction (end)

\section{Grammars}
\label{sec:grammars}
The human interface to computational devices are PLs, but the
interface to PLs is syntax.  So how to we formally define the syntax
of programming languages?  We use a mathematical device called a
grammar.  Now instead of going into the fascinating area of formal
language theory -- which is where grammars were studied extensively --
we will instead focus on several examples to learn how grammars work.

We can consider the syntax of programming languages to represent a
formal language.  A formal language is a set words generated from a
set of symbols.  Now consider the question, how can we generate all of
these words?  This is the very goal of a grammar.

Consider the set $L_{\text{even}} = \{
aa,aaaa,aaaaaa,aaaaaaaa,aaaaaaaaaa,aaaaaaaaaaaa,\ldots\}$.  This set
can be more compactly described as $\{w \,|\, w = a^{2(n+1)} \text{
  for } n \in \mathbb{N}\}$ where $\mathbb{N}$ is the set of natural
numbers, or the set $\{0,1,2,3,4,5,6,\ldots\}$.  So as we discussed
above, the set $L_{\text{even}}$ is a language, but what language?  It
is the language where all words are generated from the single symbol
$a$, and where each word contains an even number of $a$'s.  Now a
grammar for this language looks like the following:
\begin{center}
  \begin{math}
    \begin{array}{|lll|}
      \hline
      \color{red}{S} & \to & \color{blue}{a}\color{red}{E}\\
      \hline
      \color{red}{E} & \to & \color{blue}{a}\\
      \hline
      \color{red}{E} & \to & \color{blue}{a}\color{red}{O}\\
      \hline
      \color{red}{O} & \to & \color{blue}{a}\color{red}{E}\\
      \hline
    \end{array}
  \end{math}
\end{center}
A grammar consists of one or more productions denoted $A \to s$ where
we call $A$ the non-terminal of the production.  Note that the
lefthand side of a production is always a non-terminal, and not a
composition of terminals and non-terminals.  Each production generates
either a symbol in the language -- also called a non-terminal -- or is
the combination of terminals (symbols of the language) and
non-terminals.  So the righthand side of a production is a composition
of non-terminals and terminals, but it may also only consist of
non-terminals.  In the example above I have put boxes around each
production, colored the non-terminals red, and the terminals blue.
The non-terminal $S$ is a special non-terminal called the start
non-terminal.  The start non-terminal is always used first no matter
what.  A grammar may also have several productions associated with the
start non-terminal.

Once we have a grammar we can use it to generate any word in its
corresponding language.  We generate words using a grammar by
constructing a derivation.  Lets consider an example:
\begin{center}
  \begin{math}
    \begin{array}{lll}
      S & \Rightarrow & aE\\
        & \Rightarrow & aaO\\
        & \Rightarrow & aaaE\\
        & \Rightarrow & aaaaO\\
        & \Rightarrow & aaaaaE\\
        & \Rightarrow & aaaaaaO\\
        & \Rightarrow & aaaaaaaE\\
        & \Rightarrow & aaaaaaaaO\\
        & \Rightarrow & aaaaaaaaaE\\
        & \Rightarrow & aaaaaaaaaa\\
    \end{array}
  \end{math}
\end{center}
A derivation always begins with one of the productions whose lefthand
side is the start symbol.  Then we proceed by simply choosing one of
the non-terminals on the lefthand side -- it does not matter which is
chosen or in what order -- and replacing it with the lefthand side of
one of its associated productions. We repeat this until all
non-terminals have been replaced by terminals.  So every derivation
ends with a word in the language the grammar generates.

This is great, but now ask the question, what if we are given a word,
say, this
word \[aaaaaaaaaaaaaaaaaaaaaaaaaaaaaaaaaaaaaaaaaaaaaaaaaaaaaaaaa\], is
it possible to test to see if it is in the language $L_{\text{even}}$?
Now for our simple example we could just tally up the number of $a$'s
and if its even, then accept, otherwise we reject the above word, but
if this were a more complex language, like say, a PL, then that would
not work.  It turns out that we can say, well, if there exists a
derivation that ends with the above word, then it is a member of the
language $L_{\text{even}}$ otherwise it is not.  In fact, there are
known algorithms for doing just that.  In fact, this is the basic
operation for parsers, and when using a PL you obtain a syntax error,
the compiler took the input and tried to derive using the grammar of
the PL.  So grammars are the driving force behind syntax, and parser
technology.

Lets consider another more interesting example:
\begin{center}
  \begin{math}
    \begin{array}{llc}
      S & \to & ()\\      
      S & \to & (S)\\
      S & \to & SS\\      
    \end{array}
  \end{math}
\end{center}
What language does this grammar generate?  Its the language of
balanced parenthesis.  A few example words would then be $()$, $()()$,
$((((((()))))))$, and $((((((()))))))()()$.  However, these are not
words in the language generated by the above grammar: $($, $()($, and
$(((()))))$.

I would like to mention one notational convention I am using. Notice
that if I want to have multiple productions for a symbol, like $S$ in
the previous example, I write multiple productions with the same
non-terminal on the lefthand side.  There is a shorthand convention
that we will use where we combine these all into a single production.
For example, the previous grammar can be written like this:
\begin{center}
  \begin{math}
    \begin{array}{llc}
      S & \to & () \mid (S) \mid SS\\      
    \end{array}
  \end{math}
\end{center}
Here the $\mid$ symbol should be read as ``or.''

Finally, lets see how we can put all this together to formally specify
the syntax of a programming language.  The following grammar is a
grammar of a toy programming language:
\begin{center}
  \begin{math}
    \begin{array}{rll}
      (\text{booleans}) & b \to \mathsf{true} \mid \mathsf{false} \mid
      b \land b \mid b \lor b \mid \lnot b \mid e > e\\
      (\text{expressions}) & e \to 0 \mid e+1 \mid e * e \mid e + e
      \mid e - e\\
      (\text{command}) & S \to \mathsf{skip} \mid x := e \mid
      \mathsf{if}\,b\,\mathsf{then}\,S\,\mathsf{else}\,S \mid
      \mathsf{while}\,b\,\mathsf{do}\,S \\
    \end{array}
  \end{math}
\end{center}
This is slightly more complex, but the same rules we have been
discussing above hold.  An example derivation is:
\begin{center}
  \begin{math}
    \begin{array}{lllllll}
      S & \Rightarrow & \mathsf{if}\, b\,\mathsf{then}\, S \,\mathsf{else}\, S\\
      & \Rightarrow & \mathsf{if}\, e > e\,\mathsf{then}\, S \,\mathsf{else}\, S\\
      & \Rightarrow & \mathsf{if}\, e + 1 > e\, \mathsf{then}\, S \,\mathsf{else}\, S\\
      & \Rightarrow & \mathsf{if}\, 0 + 1 > e\, \mathsf{then}\, S \,\mathsf{else}\, S\\
      & \Rightarrow & \mathsf{if}\, 0 + 1 > e + 1 \,\mathsf{then}\, S \,\mathsf{else}\, S\\
      & \Rightarrow & \mathsf{if}\, 0 + 1 > e + 1 + 1 \,\mathsf{then}\, S \,\mathsf{else}\, S\\
      & \Rightarrow & \mathsf{if}\, 0 + 1 > 0 + 1 + 1 \,\mathsf{then}\, S \,\mathsf{else}\, S\\
      & \Rightarrow & \mathsf{if}\, 0 + 1 > 0 + 1 + 1 \,\mathsf{then}\, \,\mathsf{while}\, b \,\mathsf{do}\, S \,\mathsf{else}\, S\\
      & \Rightarrow & \mathsf{if}\, 0 + 1 > 0 + 1 + 1 \,\mathsf{then}\, \,\mathsf{while}\, \mathsf{true} \,\mathsf{do}\, S \,\mathsf{else}\, S\\
      & \Rightarrow & \mathsf{if}\, 0 + 1 > 0 + 1 + 1 \,\mathsf{then}\, \,\mathsf{while}\, \mathsf{true} \,\mathsf{do}\, \mathsf{skip} \,\mathsf{else}\, S\\
      & \Rightarrow & \mathsf{if}\, 0 + 1 > 0 + 1 + 1 \,\mathsf{then}\, \,\mathsf{while}\, \mathsf{true} \,\mathsf{do}\, \mathsf{skip} \,\mathsf{else}\, \mathsf{skip}\\
    \end{array}
  \end{math}
\end{center}

At this point we have seen basically all that we will need to know
about grammars for this course and the large majority of PL research.
However, the previous grammar we saw for the toy PL is a bit more
formal than we will need.  

So we now look at an alternative presentation of the same grammar:
\begin{center}
  \begin{math}
    \begin{array}{rll}
      (\text{booleans}) & b ::= \mathsf{true} \mid \mathsf{false} \mid
      b \land b \mid b \lor b \mid \lnot b \mid e > e\\
      (\text{expressions}) & e ::= 0 \mid e+1 \mid e * e \mid e + e
      \mid e - e\\
      (\text{command}) & c ::= \mathsf{skip} \mid x := e \mid
      \mathsf{if}\,b\,\mathsf{then}\,c\,\mathsf{else}\,c \mid
      \mathsf{while}\,b\,\mathsf{do}\,c \\
    \end{array}
  \end{math}
\end{center}
This is called the Backus-Naur Form (BNF) of the formal grammar above.
In PL we usually do not indicate a start non-terminal, but rather we
give the grammar as a means of understanding how to derive legal
expressions of the language.
% section grammars (end)

\section{Iffy-Lang}
\label{sec:iffy-lang}
In this section I would like to introduce the PL Iffy-Lang.  Its sytax
is as follows:
\begin{center}
  \begin{math}
    \begin{array}{lllllllll}
      \text{(Bool)}  & [[b]] & ::= & [[1]] \mid [[0]] \mid [[b1 x b2]] \mid [[b1 + b2]]
      \mid [[if b b1 b2]]\\
    \end{array}
  \end{math}
\end{center}
Now the above syntax allows us to write small programs like:
\[ [[if b 0 1]] \] 
However, the syntax does not describe how to actually compute
with such programs. So how do we run them?


The operational behavior of a program is often described by a set of
rules called the reduction rules.  The following rules describe the
operational behavior of Iffy-Lang:
\begin{center}
  \begin{mathpar}
    \ottdruleAndTrue{} \and
    \ottdruleAndFalseOne{} \and
    \ottdruleAndFalseTwo{} \and
    \ottdruleAndOne{} \and
    \ottdruleAndTwo{} \and
    \ottdruleOrTrue{} \and
    \ottdruleOrTrueTwo{} \and
    \ottdruleOrTrueOne{} \and
    \ottdruleOrFalse{} \and
    \ottdruleOrOne{} \and
    \ottdruleOrTwo{} \and
    \ottdruleIfTrue{} \and
    \ottdruleIfFalse{} \and
    \ottdruleIfOne{} \and
    \ottdruleIfTwo{} \and
    \ottdruleIfThree{}
  \end{mathpar}
\end{center}
One should read these rules from the top down as an "if-statement", that is, the rule:
\begin{center}
  \begin{math}
    $$\mprset{flushleft}
    \inferrule* [right=Name] {
      P_1
      \\
      \cdots
      \\
      P_n
    }{C}
  \end{math}
\end{center}
should be read as if $P_1, \cdots, P_n$ hold, then $C$ holds.  So for example the rule:
\[
\inferrule* [right=if] {
  [[b ~> b']]
}{[[(if b b_1 b_2) ~> (if b' b_1 b_2)]]}\\
\]
says, if $[[b ~> b']]$ holds, then $[[(if b b_1 b_2) ~> (if b' b_1 b_2)]]$ holds.

Now the symbol $\redto$ should be read as "reduces to", hence, we read
$[[b ~> b']]$ as "$[[b]]$ reduces to $[[b']]$".  The rules above can
be seen as defining what it means for a boolean $[[b]]$ to reduce to
$[[b']]$.

The previous set of inference rules tell us how to reduce an Iffy
program for one step only.  The following rules generalize this to
allow for multiple steps:
\begin{center}
  \begin{math}
    \begin{array}{lll}
      \ottdruleRefl{} &
      \ottdruleStep{} &
      \ottdruleMult{} 
    \end{array}
  \end{math}
\end{center}
The judgment $b \redto^* b'$ is called multi-step reduction.
% section iffy-lang (end)

\section{Functional Iffy}
\label{sec:functional_iffy}
The Iffy PL as it is defined in the previous section is not very
powerful.  It only allows for defining very basic boolean programs,
and does not encourage abstraction.  In this section we extend Iffy
with functions and global definitions.  This will allows us to
construct reusable functions on the booleans, and thus allowing us to
use abstraction, and even define new boolean operators that are not
currently allowed in Iffy.  Furthermore, we can define full Iffy
programs.

Adding new structures to programming languages is a huge task, and it
gets bigger as the PL gets more complicated.  We will see here that
when adding functions to Iffy we have to add the syntax, and then add
proper reduction rules capturing the operational meaning of functions.
The latter point will require some non-trivial tools.

\section{Syntax of Functional Iffy}
\label{sec:syntax_of_functional_iffy}
The first step in this extension is adding the syntax.  The syntax for
Functional Iffy is as follows:
\begin{center}
  \begin{math}
    \begin{array}{lllllllll}
      (\text{Programs}) & [[P]] & ::= & [[x := b]] \mid [[P1 NEWLINE P2]]\\
      (\text{Booleans}) & [[b]] & ::= & [[x]] \mid [[1]] \mid [[0]] \mid
           [[b1 x b2]] \mid [[b1 + b2]] \mid [[if b b1 b2]] \mid
           [[fun x => b]] \mid [[app b1 b2]]\\
    \end{array}
  \end{math}
\end{center}
This syntax gives us more control and a lot more power.  To define a
global definition we use the syntax $[[x := b]]$, where $[[x]]$ is the
name of the program we are defining, and $b$ is the actual boolean
program.  The name $[[x]]$ can be any string following the Haskell
variable naming convention.  We will use the convention that every
globally defined name must be defined before it can be used.  As an
example we can define the negation operator on booleans as follows:
\begin{center}
  \begin{math}
    \mathsf{not} := [[fun x => if x 0 1]]
  \end{math}
\end{center}
The following program defines the De Morgan's dual of conjunction:
\begin{center}
  \begin{math}
    \begin{array}{lll}
      \mathsf{not} := [[fun x => if x 0 1]]\\
      \mathsf{dmconj} := [[fun x => fun y => app not ((app not x) x
          (app not y))]]
    \end{array}
  \end{math}
\end{center}
The syntax for functions is $[[fun x => b]]$ where $[[x]]$ is the
name of the input and $[[b]]$ is the body of the function.  These are
called anonymous unary functions.  Given a function $[[b1]]$ we can
apply it to an argument using the syntax $[[app b1 b2]]$ where
$[[b2]]$ is any other program. Thus, $[[app not 0]]$ is equivalent to
$[[app (fun x => if x 0 1) 0]]$.  Function application is the locus of
computation, and is where all the fun happens during reduction.  
% section syntax_of_functional_iffy (end)

\subsection{Evaluation of Functional Iffy Programs}
\label{subsec:evaluation_of_functional_iffy_programs}
At this point we arrive at the difficult part of defining the
operational behavior of our new additions.  The question we should be
asking ourselves is how does $[[app b1 b2]]$ compute?  Suppose we have
the identity function $[[fun x => x]]$.  This function simply echos
back whatever input we give it.  So for example, $[[app (fun x => x)
    0]]$ should reduce to $[[0]]$, but how do we capture this in a
rule?

\textbf{Parse Trees.}  An extremely useful tool when thinking about
syntax of programming languages is the notion of a parse tree.  The
following definition defines how to compute the parse tree of a given
Iffy boolean program:
\vspace{10px}
\begin{definition}
  \label{def:parse-tree}
  The following meta-function defines how to compute the parse tree of
  a given boolean program by structural recursion on the input program:
  \begin{center}
      \begin{tabular}{llc}
        $[[pT 0]]$ & = & $[[0]]$\\
        $[[pT 1]]$ & = & $[[1]]$\\
        $[[pT x]]$ & = & $[[x]]$\\
        $[[pT (b1 x b2)]]$ & = & \Tree [.$\land$ {$[[pT b1]]$} {$[[pT b2]]$} ]\\\\
        $[[pT (b1 + b2)]]$ & = & \Tree [.$\lor$ {$[[pT b1]]$} {$[[pT b2]]$} ]\\\\
        $[[pT (if b b1 b2)]]$ & = & \Tree [.$\mathsf{if}$ {$[[pT b1]]$} {$[[pT b]]$} {$[[pT b2]]$} ]\\\\
        $[[pT (fun x => b)]]$ & = & \Tree [.$\mathsf{fun}\,[[x]]$ {$[[pT b]]$} ]\\\\
        $[[pT (app b1 b2)]]$ & = & \Tree [.$\mathsf{app}$ {$[[pT b1]]$} {$[[pT b2]]$} ]\\\\
      \end{tabular}
  \end{center}
\end{definition}
We call $[[pT b]]$ a meta-function because it operators outside of the
language, and is defined on programs, as opposed to being a program
itself.  This meta-function is defined by \textbf{structural
  recursion} on the input program because each recursive call is done
on a subpiece of the input program.  This style of definition is
typical of the definitions in programming language design.

\textbf{Free and bound variables.}  Consider the syntax for functions
$[[fun x => b]]$.  The body of the function, $[[b]]$, may then depend
on the input variable $[[x]]$.  This may seem like an obvious point,
but there are now some non-trivial programs that can cause confusion.
An example of one is the program $[[app (fun x => x) x]]$, and another
one is $[[fun x => fun x => x]]$.  The input variable to a function,
for example $[[x]]$ in $[[fun x => b]]$, is called a \textbf{bound
  variable} where its scope is limited to the body of the function
$[[b]]$.  One should think of these as local variables.  A bound
variable is always associated with a binder which is the
``$\mathsf{fun}$'' in the syntax of functions. Consider the first
example above, $[[app (fun x => x) x]]$, the $[[x]]$ in the body of
the function is bound, but the $[[x]]$ in the argument position is not
bound. Unbound variables are called \textbf{free variables}.  In the
second example, $[[fun x => fun x => x]]$, the $[[x]]$ in the body of
the inner function is bound to the second binder, and not the first.

Parse trees make it a lot easier to see which variables are bound, and
their associated binders.  First, construct the parse tree of a
program, and then trace the tree in reverse starting with a variable
-- the leaf of the tree -- then the first binder we hit where the
input variable matches the variable we started with is the variables
binder, and we conclude that the variable is bound, otherwise, we did
not find a binder, and we conclude it is a free variable.  For
example, consider the following parse tree:
\begin{center}
  \Tree [.$\mathsf{fun}\,x$ [ .$\mathsf{app}$ [ $[[x]]$ $[[y]]$ ] ] ]
\end{center}
If we place our finger on the $[[x]]$ leaf, and then trace backwards
through the tree we will hit $\mathsf{fun}\,[[x]]$, and then we can
conclude that $[[x]]$ is bound.  If we attempt the same procedure for
the leaf labeled $[[y]]$ we never hit a binder matching $[[y]]$, and
thus, $[[y]]$ is a free variable.  Consider a second example:
\begin{center}
  \Tree [.$\mathsf{fun}\,z$ [ .$\mathsf{app}$ [ [ .$\mathsf{app}$ [
            $[[y]]$ $[[x]]$ ] ] [ .$\mathsf{fun}\,y$ [
            .$\mathsf{fun}\,z$ $[[z]]$ ] ] ] ] ]
\end{center}
In the previous example the variables $[[y]]$ and $[[x]]$ are both
free, while the variable $[[z]]$ is bound to the inner-most binder
$\mathsf{fun}\,[[z]]$ and not the top most one.

Notice that renaming variables in a program does not change its
operational behavior.  Consider the two parse trees:
\begin{center}
  \begin{tabular}{lll}
    \Tree [.$\mathsf{fun}\,z$ [ .$\mathsf{app}$ [ [ .$\mathsf{app}$ [
            $[[y]]$ $[[x]]$ ] ] [ .$\mathsf{fun}\,y$ [
              .$\mathsf{fun}\,z$ $[[z]]$ ] ] ] ] ]
    & \quad &
    \Tree [.$\mathsf{fun}\,x$ [ .$\mathsf{app}$ [ [ .$\mathsf{app}$ [
            $[[z]]$ $[[w]]$ ] ] [ .$\mathsf{fun}\,z$ [
            .$\mathsf{fun}\,x$ $[[x]]$ ] ] ] ] ]
  \end{tabular}
\end{center}
These two denote the exact same program.  We say that they are
\textbf{equivalent up to renaming of variables}.  The process of
renaming variables is called $\alpha$-conversion.

Keep in mind that when renaming variables the status of the varible --
bound or free -- must be the same.  As an example the following two
programs are \textbf{not the same}:
\begin{center}
  \begin{tabular}{lll}
    \Tree [.$\mathsf{fun}\,x$ [ .$\mathsf{app}$ [ $[[x]]$ $[[y]]$ ] ]
    ]
    & \quad &
    \Tree [.$\mathsf{fun}\,x$ [ .$\mathsf{app}$ [ $[[x]]$ $[[x]]$ ] ] ]
  \end{tabular}
\end{center}
At this point we can define a bound variable as a variable underneath
a binder labeled with its name, otherwise we call the variable bound.

There is an alternative way of determining when a variable is bound or
free by defining the set of all free variables of a program:
\begin{center}
  \begin{math}
    \begin{array}{lll}
      [[FV(0)]] & = & \emptyset\\
      [[FV(1)]] & = & \emptyset\\
      [[FV(x)]] & = & \{[[x]]\}\\
      [[FV(b1 x b2)]] & = & [[FV(b1)]] \cup [[FV(b2)]]\\
      [[FV(b1 + b2)]] & = & [[FV(b1)]] \cup [[FV(b2)]]\\
      [[FV(if b b1 b2)]] & = & [[FV(b)]] \cup [[FV(b1)]] \cup [[FV(b2)]]\\
      [[FV(fun x => b)]] & = & [[FV(b)]] - \{x\}\\
      [[FV(app b1 b2)]] & = & [[FV(b1)]] \cup [[FV(b2)]]\\
    \end{array}
  \end{math}
\end{center}
Then a variable $[[x]]$ is free in the program $[[b]]$ iff $[[x]]\in
[[FV(b)]]$. Otherwise we say it is bound.

Similarly, we can construct the set of all bound variables:
\begin{center}
  \begin{math}
    \begin{array}{lll}
      [[BV(0)]] & = & \emptyset\\
      [[BV(1)]] & = & \emptyset\\
      [[BV(x)]] & = & \emptyset\\
      [[BV(b1 x b2)]] & = & [[BV(b1)]] \cup [[BV(b2)]]\\
      [[BV(b1 + b2)]] & = & [[BV(b1)]] \cup [[BV(b2)]]\\
      [[BV(if b b1 b2)]] & = & [[BV(b)]] \cup [[BV(b1)]] \cup [[BV(b2)]]\\
      [[BV(fun x => b)]] & = & [[BV(b)]] \cup \{x\}\\
      [[BV(app b1 b2)]] & = & [[BV(b1)]] \cup [[BV(b2)]]\\
    \end{array}
  \end{math}
\end{center}
Now a variable $[[x]]$ is bound in a program $[[b]]$ iff $[[x]] \in
[[BV(b)]]$.  Otherwise we say it is free.

\textbf{Substitution.}  Our ultimate goal is to be able to define the
evaluation rules for function application.  That is, if we have an
expression $[[app (fun x => x) 0]]$ how to we define the evaluation
rules so that this expression evaluates to the expression $[[0]]$?

To answer this question we need to define a new meta-function called
\textbf{capture-avoiding substitution}.  This is an operation that
takes as input two Iffy programs, say $[[b1]]$ and $[[b2]]$, and the
name of a free variable, say $[[x]]$, that may or not be free in
$[[b2]]$, and then outputs a new Iffy program where every occurrence
of $[[x]]$ in $[[b2]]$ is replaced by $[[b1]]$.  Consider an example,
an Iffy program in parse-tree form is defined as follows:
\begin{center}
      \Tree [.$\mathsf{app}$ [ $[[x]]$ [ .$\mathsf{fun}\,z$ $[[x]]$
            $[[y]]$ ] ] ]
\end{center}
Then substituting the program $[[1]]$ for $[[x]]$ in the above program
yields:
\begin{center}
      \Tree [.$\mathsf{app}$ [ $1$ [ .$\mathsf{fun}\,z$ $1$
            $[[y]]$ ] ] ]
\end{center}
If we replace $[[x]]$ with program $[[0 x 1]]$ then we obtain the
following:
\begin{center}
      \Tree [.$\mathsf{app}$ [ [ .$\land$ $0$ $1$ ] [ .$\mathsf{fun}\,z$ [ .$\land$ $0$ $1$ ]    $[[y]]$ ] ] ]
\end{center}

Now if we replace $[[x]]$ with the program $[[app z 0]]$, note that
$[[z]]$ is a free variable, we obtain the following:
\begin{center}
      \Tree [.$\mathsf{app}$ [ [ .$\mathsf{app}$ $z$ $0$ ] [ .$\mathsf{fun}\,z$ [ .$\mathsf{app}$ $z$ $0$ ]    $[[y]]$ ] ] ]
\end{center}
This does not seem quite right, because the variable $[[z]]$ was free
in the program we were replacing $[[x]]$ with, and so it should always
remain free after the substitution, but at the second location of
$[[x]]$ the variable $[[z]]$ has become bound.  This is called
\textbf{variable capture} and is something that we must prevent.

To prevent variable capture first rename all the variables in $[[b1]]$
so that they are distinct from the variables in $[[b2]]$.  This is
called \textbf{$\alpha$-conversion}.  Then do the substitution.  So if
we replace $[[x]]$ with the program $[[app z 0]]$, we first change
$[[z]]$ to a new name, say $[[y]]$, resulting in $[[app y 0]]$, and
then do the substitution obtaining:
\begin{center}
      \Tree [.$\mathsf{app}$ [ [ .$\mathsf{app}$ $y$ $0$ ] [ .$\mathsf{fun}\,z$ [ .$\mathsf{app}$ $y$ $0$ ]    $[[y]]$ ] ] ]
\end{center}
We can see that every occurrence of $[[y]]$ is free in the final
result.

At this point we define the substitution function on programs.
\begin{definition}
  \label{def:substitution}
  Capture-avoiding substitution is defined as follows where it is
  assumed that the variable names in $[[b1]]$ are distinct from the
  variable names in $[[b2]]$.
  \begin{center}
    \begin{math}
      \begin{array}{lll}
        [[ [b1/x]0]] & = & [[0]]\\
        [[ [b1/x]1]] & = & [[1]]\\
        [[ [b1/x]x]] & = & [[b1]]\\
        [[ [b1/x]y]] & = & [[y]]\\
        \,\,\,\,\,\,\,\text{ where } x \neq y\\
        [[ [b1/x](b2 x b3)]] & = & [[([b1/x]b2) x ([b1/x]b3)]]\\
        [[ [b1/x](b2 + b3)]] & = & [[([b1/x]b2) + ([b1/x]b3)]]\\\\
        [[ [b1/x](if b b2 b3)]] & = & [[if ([b1/x]b) ([b1/x]b2) ([b1/x]b3)]]\\
        [[ [b1/x](fun x => b2)]] & = & [[fun x => b2]]\\
        [[ [b1/x](fun y => b2)]] & = & [[fun y => [b1/x]b2]]\\
        \,\,\,\,\,\,\,\text{ where } x \neq y\\        
      \end{array}
    \end{math}
  \end{center}
\end{definition}
Here is an example of using this function:
\begin{center}
  \begin{math}
    \begin{array}{lll}
      [[ [0/x](fun y => if x 1 (x + y))]]
      & = &
      [[ fun y => [0/x](if x 1 (x + y))]]\\
      & = &
      [[ fun y => if ([0/x]x) ([0/x]1) ([0/x](x + y))]]\\
      & = &
      [[ fun y => if 0 ([0/x]1) ([0/x](x + y))]]\\
      & = &
      [[ fun y => if 0 1 ([0/x](x + y))]]\\
      & = &
      [[ fun y => if 0 1 (([0/x]x) + ([0/x]y))]]\\
      & = &
      [[ fun y => if 0 1 (0 + ([0/x]y))]]\\
      & = &
      [[ fun y => if 0 1 (0 + y)]]\\   
    \end{array}
  \end{math}
\end{center}


\textbf{Evaluation.}  We have finally arrived at the main event
evaluation of functional Iffy!  Suppose we were asked to run the Iffy
program $[[app (fun x => x) 0]]$, how would we do this?  It turns out
that a special rule called the $\beta$-rule can be defined that allows
us to simplify a program of the form $[[app (fun x => b) b']]$ for any
programs $[[b]]$ and $[[b']]$.  A program of this form is called a
\textbf{$\beta$-reducible expression ($\beta$-redex or redex)}.

To simplify a $\beta$-redex we will use capture-avoiding substitution.
The following inference rule defines evaluation for $\beta$-redexes:
\[ \ottdruleBeta{} \]
This rule says that in one step we can evaluate a $\beta$-redex into a
substitution where we replace the variable, $[[x]]$, in $[[b]]$ with
the argument the function is being applied to $[[b']]$.  Thus,
function application is just substitution or replacement.  Note that
on the left-hand side $[[x]]$ is bound in $[[b]]$, but when we move to
the right-hand side we rip out the application and the binder, and
thus, $[[x]]$ changes from a bound variable to a free variable which
can be replaced using substitution.  In this rule we write
$[[ [b'/x]b]]$ because we do not yet know what $[[b']]$ and $[[b]]$
are, but when evaluating a particular example we will know what they
are, and thus, $[[ [b/x]b]]$ should be replaced with the actual output
of the function.

We can now state the final single-step evaluation rules for functional
Iffy:
\begin{center}
  \begin{mathpar}
    \ottdruleAndTrue{} \and
    \ottdruleAndFalseOne{} \and
    \ottdruleAndFalseTwo{} \and
    \ottdruleAndFalse{} \and
    \ottdruleAndOne{} \and
    \ottdruleAndTwo{} \and
    \ottdruleOrTrue{} \and
    \ottdruleOrTrueTwo{} \and
    \ottdruleOrTrueOne{} \and
    \ottdruleOrFalse{} \and
    \ottdruleOrOne{} \and
    \ottdruleOrTwo{} \and
    \ottdruleIfTrue{} \and
    \ottdruleIfFalse{} \and
    \ottdruleIfOne{} \and
    \ottdruleIfTwo{} \and
    \ottdruleIfThree{} \and
    \ottdruleBeta{} \and
    \ottdruleFun{} \and
    \ottdruleAppOne{} \and 
    \ottdruleAppTwo{}
  \end{mathpar}
\end{center}
The multiset reduction rules do not change.  They remain completely
the same.

We now give several example derivations:
\begin{example}
  \label{ex:eval1}
  \ \\
  \begin{center}
  \begin{math}
    $$\mprset{flushleft}
    \inferrule* [right=Beta] {
      \,
    }{[[app (fun x => x) 0 ~> 0]]}
  \end{math}
\end{center}
The previous example is a complete and valid derivation.  Why is this?
It is valid, because $[[ [0/x]x]] = [[0]]$, and thus, the previous
derivation applies the $\beta$-rule.
\end{example}

\begin{example}
  \label{ex:eval2}
  \ \\
  \begin{center}
  \begin{math}
    $$\mprset{flushleft}
    \inferrule* [right=Mult] {
      $$\mprset{flushleft}
      \inferrule* [right=Step] {
        $$\mprset{flushleft}
      \inferrule* [right=Beta] {
        \,
      }{[[app (fun x => if x 0 1) (0 x 1) ~> if (0 x 1) 0 1]]}
      }{[[app (fun x => if x 0 1) (0 x 1) ~>* if (0 x 1) 0 1]]}
      \\
      D
    }{[[app (fun x => if x 0 1) (0 x 1) ~>* 1]]}
  \end{math}  
  \end{center}
  \ \\D:\\
  \begin{center}
    \begin{math}
      $$\mprset{flushleft}
  \inferrule* [right=Mult] {
    $$\mprset{flushleft}
    \inferrule* [right=Step] {
      $$\mprset{flushleft}
      \inferrule* [right=If1] {
        $$\mprset{flushleft}
        \inferrule* [right=AndFalse1] {
          \,
        }{[[0 x 1 ~> 0]]}
      }{[[if (0 x 1) 0 1 ~> if 0 0 1]]}
    }{[[if (0 x 1) 0 1 ~>* if 0 0 1]]}
    \\
      $$\mprset{flushleft}
    \inferrule* [right=IfFalse] {
      \,
    }{[[if 0 0 1 ~>* 1]]}
  }{[[if (0 x 1) 0 1 ~>* 1]]}
    \end{math}
  \end{center}
  Note that $[[ [0 x 1/x](if x 0 1)]] = [[if (0 x 1) 0 1]]$.
\end{example}

\begin{example}
  \label{ex:eval3}
  \ \\
  \begin{center}
    \scriptsize
    \begin{math}
      $$\mprset{flushleft}
      \inferrule* [right={\scriptsize Mult}] {
        $$\mprset{flushleft}
        \inferrule* [right={\scriptsize Step}] {
          $$\mprset{flushleft}
          \inferrule* [right={\scriptsize App1}] {
            $$\mprset{flushleft}
            \inferrule* [right={\scriptsize Beta}] {
              \,
            }{[[app (fun x => fun y => if x (x x y) (x + y)) 0 ~> fun y => if 0 (0 x y) (0 + y)]]}
          }{[[app (app (fun x => fun y => if x (x x y) (x + y)) 0) 1 ~> app (fun y => if 0 (0 x y) (0 + y)) 1]]}
        }{[[app (app (fun x => fun y => if x (x x y) (x + y)) 0) 1 ~>* app (fun y => if 0 (0 x y) (0 + y)) 1]]}
        \\
        D_1
      }{[[app (app (fun x => fun y => if x (x x y) (x + y)) 0) 1 ~>* 1]]}
    \end{math}
  \end{center}

  \ \\$D_1$:\\
  \begin{center}
    \scriptsize
    \begin{math}
      $$\mprset{flushleft}
      \inferrule* [right={\scriptsize Mult}] {
        $$\mprset{flushleft}
        \inferrule* [right={\scriptsize Step}] {
          $$\mprset{flushleft}
          \inferrule* [right={\scriptsize Beta}] {
            \,
          }{[[app (fun y => if 0 (0 x y) (0 + y)) 1 ~> if 0 (0 x 1) (0 + 1)]]}
        }{[[app (fun y => if 0 (0 x y) (0 + y)) 1 ~>* if 0 (0 x 1) (0 + 1)]]}
        \\
        D_2
      }{[[app (fun y => if 0 (0 x y) (0 + y)) 1 ~>* 1]]}
    \end{math}
  \end{center}

    \ \\$D_2$:\\
  \begin{center}
    \scriptsize
    \begin{math}
      $$\mprset{flushleft}
      \inferrule* [right={\scriptsize Mult}] {
        $$\mprset{flushleft}
        \inferrule* [right={\scriptsize Step}] {
          $$\mprset{flushleft}
          \inferrule* [right={\scriptsize IfFalse}] {
            \,
          }{[[if 0 (0 x 1) (0 + 1) ~> 0 + 1]]}
        }{[[if 0 (0 x 1) (0 + 1) ~>* 0 + 1]]}
        \\
          $$\mprset{flushleft}
        \inferrule* [right={\scriptsize Step}] {
          $$\mprset{flushleft}
          \inferrule* [right={\scriptsize orTrue2}] {
            \,
          }{[[0 + 1 ~> 1]]}
        }{[[0 + 1 ~>* 1]]}
      }{[[if 0 (0 x 1) (0 + 1) ~>* 1]]}
    \end{math}
  \end{center}
\end{example}

\textbf{Rewrite Rules.}  So far we have used the evaluation rules to
show that one program evaluates into another.  Is it possible to
evaluate the program without knowing what the output would be?  That
is, given a program $[[b]]$ is it possible to simply run it?  Yes, we
can, and we can define rules, called \textbf{rewrite rules}, that
allow us to do this.

The set of rewrite rules for functional Iffy are defined as follows:
  \begin{mathpar}
      \inferrule* [right=AndTrue] {
        \,
      }{[[(1 x 1) ~>* 1]]}
      \and
      \inferrule* [right=AndFalse1] {
        \,
      }{[[(0 x 1) ~>* 0]]}
      \and
      \inferrule* [right=AndFalse2] {
        \,
      }{[[(1 x 0) ~>* 0]]}
      \and
      \inferrule* [right=AndFalse] {
        \,
      }{[[(0 x 0) ~>* 0]]}
      \and
      \inferrule* [right=OrTrue] {
        \,
      }{[[(1 + 1) ~>* 1]]}
      \and
      \inferrule* [right=OrTrue1] {
        \,
      }{[[(1 + 0) ~>* 1]]}
      \and
      \inferrule* [right=OrTrue1] {
        \,
      }{[[(0 + 1) ~>* 1]]}
      \and
      \inferrule* [right=OrFalse] {
        \,
      }{[[(0 + 0) ~>* 0]]}
      \and
      \inferrule* [right=IfTrue] {
        \,
      }{[[if 1 b1 b2 ~>* b1]]}
      \and
      \inferrule* [right=IfFalse] {
        \,
      }{[[if 0 b1 b2 ~>* b2]]}
      \and
      \inferrule* [right=Beta] {
        \,
      }{[[app (fun x => b) b' ~>* [b'/x]b]]}      
  \end{mathpar}
They are the same as all the axioms of the single-step reduction
rules.  We will use these rules to pattern match on subprograms of
larger programs against the left-hand side of the rules above, and
then replace the matching subprogram with the right-hand side.
However, we first must understand what a subprogram.
\vspace{10px}
\begin{definition}
  \label{def:subprog}
  A program $[[b1]]$ is a subprogram of $[[b2]]$ if and only if at
  least one of the following is true:
  \begin{itemize}
  \item $[[b1]]$ is literally $[[b2]]$
  \item If $[[b2]] = [[b3 x b4]]$ for some programs $[[b3]]$ and 
    $[[b4]]$, and $[[b1]]$ is one of these two programs
  \item If $[[b2]] = [[b3 + b4]]$ for some programs $[[b3]]$ and 
    $[[b4]]$, and $[[b1]]$ is one of these two programs
  \item If $[[b2]] = [[if b3 b4 b5]]$ for some programs $[[b3]]$,
    $[[b4]]$, and $[[b5]]$, and $[[b1]]$ is one of these three
    programs
  \item If $[[b2]] = [[fun x => b3]]$ for some program $[[b3]]$, and
    $[[b1]]$ is $[[b3]]$
  \item If $[[b2]] = [[app b3 b4]]$ for some programs $[[b3]]$ and 
    $[[b4]]$, and $[[b1]]$ is one of these two programs
  \end{itemize}
\end{definition}
The program $[[0]]$ is a subprogram of $[[0]]$, $[[0 + x]]$ is a
subprogram of $[[fun x => (0 + x)]]$.  Similarly, $[[app y (fun x =>
    x)]]$ is a subprogram of $[[0 x (app y (fun x => x))]]$.   Some
none subprograms would be $\mathsf{app}$ and $\mathsf{fun}\,[[x]]$
which are not subprograms of any programs.

Now if a subprogram matches the left-hand side of the set of axioms
listed above, then we are allowed to replace that subprogram with the
right-hand side.  An example will best illustrate how we are going to
write this down (then we will discuss this more in class):
\begin{center}
  \begin{math}
    \begin{array}{rcl}
      [[app (ul(app (fun x => fun y => if x (x x y) (x + y)) 0)) 1]]
      & \stackrel{\textsc{Beta}}{\redto^*} & 
      [[ul(app (fun y => if 0 (0 x y) (0 + y)) 1)]]\\
      & \stackrel{\textsc{Beta}}{\redto^*} & 
      [[ul(if 0 (0 x 1) (0 + 1))]]\\
      & \stackrel{\textsc{IfFalse}}{\redto^*} & 
      [[ul(0 + 1)]]\\
      & \stackrel{\textsc{OrTrue2}}{\redto^*} & 
      [[1]]
    \end{array}
  \end{math}
\end{center}
First, notice how much shorter these are, but also note that the order
in which we simplified the above does not matter.  For example, we
could have done the following:
\begin{center}
  \begin{math}
    \begin{array}{rcl}
      [[app (ul(app (fun x => fun y => if x (x x y) (x + y)) 0)) 1]]
      & \stackrel{\textsc{Beta}}{\redto^*} & 
      [[app (fun y => ul(if 0 (0 x y) (0 + y))) 1]]\\
      & \stackrel{\textsc{IfFalse}}{\redto^*} & 
      [[ul(app (fun y => (0 + y)) 1)]]\\
      & \stackrel{\textsc{Beta}}{\redto^*} & 
      [[ul(0 + 1)]]\\
      & \stackrel{\textsc{OrTrue2}}{\redto^*} & 
      [[1]]\\      
    \end{array}
  \end{math}
\end{center}
When rewriting instead of deriving the format above must be adhered
to on homework assignments.
% subsection evaluation_of_functional_iffy_programs (end)
% section functional_iffy (end)


\bibliographystyle{plain}
\bibliography{thesis}
\end{document}
